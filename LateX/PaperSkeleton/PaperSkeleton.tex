% Documentclass
%%%%%%%%%%%%%%%%%%%%%%%%%%%%%%%%%%%%%%%%%%%%%%%%%%%%%%%%%%%%%%%%%%%%%%%%%%%%%%%%%%%%%%%%%%%%%%%%%%%%%%%%%%%%%%%%%%%
\documentclass{aa}

% Packages
%%%%%%%%%%%%%%%%%%%%%%%%%%%%%%%%%%%%%%%%%%%%%%%%%%%%%%%%%%%%%%%%%%%%%%%%%%%%%%%%%%%%%%%%%%%%%%%%%%%%%%%%%%%%%%%%%%%
% Document layout
\usepackage[breaklinks=true]{hyperref}
\usepackage[varg]{txfonts}
% Bib
\usepackage{natbib}
% Maths
\usepackage{amsmath}
% Astronomy
\usepackage{wasysym}


% New commands
%%%%%%%%%%%%%%%%%%%%%%%%%%%%%%%%%%%%%%%%%%%%%%%%%%%%%%%%%%%%%%%%%%%%%%%%%%%%%%%%%%%%%%%%%%%%%%%%%%%%%%%%%%%%%%%%%%%
% Units
\newcommand{\Msun}{\,$M_{\astrosun}$}
\newcommand{\m}{\textmu}
\newcommand{\umum}{\,\m$\mathrm{m}$}

%%%%%%%%%%%%%%%%%%%%%%%%%%%%%%%%%%%%%%%%%%%%%%%%%%%%%%%%%%%%%%%%%%%%%%%%%%%%%%%%%%%%%%%%%%%%%%%%%%%%%%%%%%%%%%%%%%%
\begin{document}
\title{Example text\\ that is split}
\subtitle{II.\@ An example text with infinitesimal scientific value\\
whose title and subtitle may also be split}
\author{M. B. Scheuck
    \and R. van Boekel
\and Th. Henning}
\institute{Max Planck Institute for Astronomy, Königstuhl 17, D-69117 Heidelberg, Germany} \date{In preparation}
\abstract{} {Here the aims go} {Here the methods go.} {Here the results go} {Here the conclusions go}
\keywords{stars: pre-main sequence -- planetary systems: protoplanetary disks
-- stars: imaging stars -- individual object: HD\,142666}
\maketitle

%%%%%%%%%%%%%%%%%%%%%%%%%%%%%%%%%%%%%%%%%%%%%%%%%%%%%%%%%%%%%%%%%%%%%%%%%%%%%%%%%%%%%%%%%%%%%%%%%%%%%%%%%%%%%%%%%%%
\section{Introduction}
% Maybe add object parameters from MIDI and else, put them in somewhere?
Pre-main sequence (PMS) objects are characterized by observational features of
the early evolutional stage. These features are emission lines, infrared- and/or
ultraviolet excesses and brightness variations. There are two distinct classes
of such PMS objects: The Herbig Ae/Be- and the T Tauri stars (TTS), whereby
Herbig Ae/Be stars are objets of intermediate mass to massive (1.5 to 4\Msun/4 to
10\Msun, for Ae and Be, respectively) and TTS are low mass (<
1\Msun). With Herbig Ae crossing the instability region of the Hertzsprung Russel
(HR) diagram in their evolution\citet{Zwintz2008}.
Around the stars there are various classifications of disks: They are based on
the spectral energy distribution (SED) in the mid-/far-infrared (IR) excess,
which - for intermediate-mass stars - can either be fit by a power-law
continuum or requires an additonal blackbody and can be split into
Group I and II (GI/GII). At first, these were believed to be clearly
split into distinct geometries (flat- and flared disks), but in GII there are\citet{Garufi2017}.
% Complete it and add classification of GI, GII disks properties
%%%%%%%%%%%%%%%%%%%%%%%%%%%%%%%%%%%%%%%%%%%%%%%%%%%%%%%%%%%%%%%%%%%%%%%%%%%%%%%%%%%%%%%%%%%%%%%%%%%%%%%%%%%%%%%%%%%
\section{Target description}
In this paper, we investigate the object HD\,142666, belonging to the $\rho$
Ophiuchi star formation region\citet{Schegerer2013}. It shows
irregular brightness variations and colour changes, the star becomes redder when
fainter, which can be explained by looking edge-on onto the circumstellar dust
disk, with grain-sizes of 1\umum. In the line of sight, the dust clouds obscure
the star and cause the reddening, which is a typical phenomenon for Herbig
Ae/Be stars called `non-periodic Algol-like brightness minima/UX Ori type
variations'. The light of the star is always diluted by its dusty
environment, which makes a precise determination of the absolute magnitude and
luminosity difficult. Furthermore, HD\,142666 is a type II Herbig Ae star
(showing a double-peaked $H_{\alpha}$ emission profile)\citet{Zwintz2008}.
HD\,142666 is an isolated A3/A8 star with an inner radius slightly larger than
the dust sublimation radius at sub-AU scale with its disk being a prototype of group II (GII)
disks\citet{Garufi2017}.
%%%%%%%%%%%%%%%%%%%%%%%%%%%%%%%%%%%%%%%%%%%%%%%%%%%%%%%%%%%%%%%%%%%%%%%%%%%%%%%%%%%%%%%%%%%%%%%%%%%%%%%%%%%%%%%%%%%
\section{Observations and data reduction}
HD\,142666, an intermediate-mass Herbig Ae star has been studied quite frequently already
(e.g., observations done with IOTA (), AMBER, and especially MIDI (), in case
of MIDI it was operated in N-band ($8-12$\umum)\citet{Schegerer2013}).
Multi-Apterture mid-Infrared SpectroScopic Experiment (MATISSE) observations of
HD14266 in L/M- and N-band were performed on several occasions from March to June 2019.
Two arrays, the Unit Telescopes (UT) and the Auxiliary Telescopes (AT) of the Very Large Telescopic Interferometer (VLTI) of the European Southern Observatory (ESO) in Paranal, Chile were used.
For the UTs there is only one configuration, which is of interest for the N-band observations.
However, for the ATs there are several configurations that have been used (small, medium, large and extended).
% Add the max resolution of the configurations as well as the flux and such
Calibration and editing of the $\left(\textit{u, v}\right)$ data was performed using standard techniques within the MATISSE data pipeline and the common pipeline library, the ESO Recipe Execution Tool (EsoRex).
\subsection{Data reduction of the N-band data}
The data was reduced with the MATISSE Data Reduction Software (DRS) using the
1.5.1 version of the pipeline.
%%%%%%%%%%%%%%%%%%%%%%%%%%%%%%%%%%%%%%%%%%%%%%%%%%%%%%%%%%%%%%%%%%%%%%%%%%%%%%%%%%%%%%%%%%%%%%%%%%%%%%%%%%%%%%%%%%%
\section{Model geometry}
%%%%%%%%%%%%%%%%%%%%%%%%%%%%%%%%%%%%%%%%%%%%%%%%%%%%%%%%%%%%%%%%%%%%%%%%%%%%%%%%%%%%%%%%%%%%%%%%%%%%%%%%%%%%%%%%%%%
\section{Results}
%%%%%%%%%%%%%%%%%%%%%%%%%%%%%%%%%%%%%%%%%%%%%%%%%%%%%%%%%%%%%%%%%%%%%%%%%%%%%%%%%%%%%%%%%%%%%%%%%%%%%%%%%%%%%%%%%%%
\section{Discussion}
%%%%%%%%%%%%%%%%%%%%%%%%%%%%%%%%%%%%%%%%%%%%%%%%%%%%%%%%%%%%%%%%%%%%%%%%%%%%%%%%%%%%%%%%%%%%%%%%%%%%%%%%%%%%%%%%%%%
\section{Summary and conclusions}
%%%%%%%%%%%%%%%%%%%%%%%%%%%%%%%%%%%%%%%%%%%%%%%%%%%%%%%%%%%%%%%%%%%%%%%%%%%%%%%%%%%%%%%%%%%%%%%%%%%%%%%%%%%%%%%%%%%
\section*{References}
%%%%%%%%%%%%%%%%%%%%%%%%%%%%%%%%%%%%%%%%%%%%%%%%%%%%%%%%%%%%%%%%%%%%%%%%%%%%%%%%%%%%%%%%%%%%%%%%%%%%%%%%%%%%%%%%%%%

\bibliographystyle{aa}
\bibliography{hd142666}

\end{document}
% Are there ALMA or MIDI-datasets? Make plots
